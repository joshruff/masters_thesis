%%%%%%%%%%%%%%%%%%%%%%%%%%%%%%%%%%%%%%%%%%%%%%%%%%%
%
%  New template code for TAMU Theses and Dissertations starting Fall 2016.  
%
%
%  Author: Sean Zachary Roberson
%  Version 3.17.09
%  Last Updated: 9/21/2017
%
%%%%%%%%%%%%%%%%%%%%%%%%%%%%%%%%%%%%%%%%%%%%%%%%%%%
%%%%%%%%%%%%%%%%%%%%%%%%%%%%%%%%%%%%%%%%%%%%%%%%%%%%%%%%%%%%%%%%%%%%%
%%                           ABSTRACT 
%%%%%%%%%%%%%%%%%%%%%%%%%%%%%%%%%%%%%%%%%%%%%%%%%%%%%%%%%%%%%%%%%%%%%

\chapter*{ABSTRACT}
\addcontentsline{toc}{chapter}{ABSTRACT} % Needs to be set to part, so the TOC doesnt add 'CHAPTER ' prefix in the TOC.

\pagestyle{plain} % No headers, just page numbers
\pagenumbering{roman} % Roman numerals
\setcounter{page}{2}

Avionics in modern aircraft have multiple redundant wiring paths in case of failure. The aerospace industry acquired spectrum for wireless avionics which would reduce the amount of necessary wiring, but must prove compatibility with the radio altimeters incumbent to the band for certification. This work covers the development of a reference test bed validated by radio altimeter and aircraft manufacturers. This test bed was automated in a modular framework which allowed the rapid modification of software to suit a wide variety of test conditions. This work also covers the three altimeter testing regimens which used this test bed, and the development of reporting formats which supported the creation of international standards based on these results. 
 

\pagebreak{}
