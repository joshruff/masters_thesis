%%%%%%%%%%%%%%%%%%%%%%%%%%%%%%%%%%%%%%%%%%%%%%%%%%%
%
%  New template code for TAMU Theses and Dissertations starting Fall 2016.  
%
%
%  Author: Sean Zachary Roberson
%  Version 3.17.09
%  Last Updated: 9/21/2017
%
%%%%%%%%%%%%%%%%%%%%%%%%%%%%%%%%%%%%%%%%%%%%%%%%%%%

%%%%%%%%%%%%%%%%%%%%%%%%%%%%%%%%%%%%%%%%%%%%%%%%%%%%%%%%%%%%%%%%%%%%%%%
%%%                           SECTION II
%%%%%%%%%%%%%%%%%%%%%%%%%%%%%%%%%%%%%%%%%%%%%%%%%%%%%%%%%%%%%%%%%%%%%%


\chapter{METHODS}
\section{Basic Altimeter Test Bed Setup}
In addition to laying out performance standards for radar altimeters, DO-155 standards ~\cite{noauthor_minimum_1974} also specify a basic test setup for verifying an altimeter is functioning properly. The standards also elaborated on the necessary characteristics of the altitude simulator. 

The simulator would need to ''consist of variable and fixed RF attenuators''~\cite{noauthor_minimum_1974}  to simulate the loop loss an altimeter experiences aboard an aircraft (see section 1.6.3). The simulator would also need a length of ''coaxial cables or other suitable delays''~\cite{noauthor_minimum_1974}  to simulate the physical time delay experienced by an altimeter signal between the transmitter and receiver (see section 1.6.2). The attenuated and delayed RF energy from the transmitter would then need to be fed back into the receiver. 
\begin{figure}[ht]
\centering
\includegraphics[scale=0.5]{DO-155_Test_Setup.PNG}
\caption[]{Basic Altimeter Test Setup from DO-155~\cite{noauthor_minimum_1974}}

\label{fig:Basic Testbed}

\end{figure}
Additionally, the standards specified that any test equipment must account for cross coupling between transmitting and receiving antennas. DO-155 emphasized that the simulator achieve the desired altitude within 1\% and the correct attenuation within 2.5dB~\cite{noauthor_minimum_1974}.

\section{Modified Altimeter Test Setup} 
A modified version of the Altimeter test setup specified by DO-155 was implemented in our lab. The modifications were designed to inject interference into the line after the altimeter signal had passed through the altitude simulator. 
\subsection{Reading the Altimeter Output}
The altimeter outputs height data on an ARINC 429 cable setup. Our test setup uses a Ballard 429 to read the height data, as well as time stamps and any warnings associated with each data point. Ballard CoPilot Software provides a display which allows the real time monitoring of altimeter output as well as the easy export of test data to Microsoft Excel documents for post processing. 
\subsection{Implementing the Altitude Simulator}
\subsubsection{Time Delay}
For these tests, spools of fiber optic cables created a time delay. The RF output from the altimeter transmitter was fed by coax connection to the fiber optic transceiver, which could either pass the signal to a single fiber optic spool or a series of cascaded spools to achieve a desired height. This setup contained optical spools of 500, 1000, 2000, and 4500 feet, each of which could be used individually or in conjunction with any or all of the other spools to implement a delay.

 The optical transceiver and cascaded spools also contribute an attenuation to the loop loss which varies based on the number of spools cascaded. A single spool setup has an attenuation verified experimentally to be 29dB, with an additional 2dB loss added for each additional cascaded spool.

Later tests would modify this delay setup to test an altimeter in takeoff and landing scenarios. The much lower height in these scenarios meant that a spool of coax could be used tor the delay instead of fiber optic cables. Two coax spools provided a height of 40 ft and 100 ft for testing these scenarios, with a 6dB and 36dB attenuation contributed to the loop loss respectively.  

\subsubsection{Achieving Standard Loop Losses}



\subsection{Generating Interference Signals}

\section{Python Test Software}




\subsection{Test Main Loop}
\subsection{Post - Processing}
\subsubsection{Parsing The Copilot Log}
\subsubsection{Mapping Interference Signals to Copilot Data}
\subsubsection{Plotting the Data}

\section{Setup for WAIC plus Altimeter Interference}

\subsection{Calibrating the VCOs}

\section{Setup for Testing Wider Bandwidth Interference}

\subsection{Handling Multiple Vector Signal Generators}

%\section{S}

%
%Figure (and table) titles should be consistent through the document. All captions should be placed either above or below the object it describes. This is done by placing the \textit{caption} in the correct place. While continued figures are allowed by the Thesis Manual, it is not suggested that any continued figures be included in a \LaTeX\ document. The figure below is from Linux Mint, showing a portion of a desktop.
%
%\begin{figure}[H]
%	\centering
%	\includegraphics[width = 5.75in]{Desktop.png}
%	\caption{A typical desktop space in Linux Mint.}
%\end{figure}
%
%The figure below is taken from R. While there are packages available to import graphics from R, MATLAB, and similar software, it is probably best to export plots generated by these programs as a PNG file, and then import it via the \textit{includegraphics} command.
%
%\begin{figure}[H]
%	\centering
%	\includegraphics[scale=0.55]{UnemDiffACF.png}
%	\singlespace
%	\caption{The autocorrelation function (ACF) of the differenced unemployment series. Seasonal adjustments may be needed.}
%\end{figure}
%
%It is highly suggested that you scale the figures so that they fit within the margins. Almost all the figures included in this document for the sake of example have been scaled. It is best to use PNG and JPEG files as figures.
%
%The last figure here is a screenshot from the Linux terminal.
%
%\begin{figure}[H]
%	\centering
%	\includegraphics[width=4.75in]{Terminal1.png}
%	\caption{The Linux terminal. The commands shown are from a two-dimensional mesh generator that triangulates a domain in the plane. Files containing nodes, elements, the polygon, and the edges are created.}
%\end{figure}
%
%\section{Table Placement, Size and Table Title}
%
%Here is a table, displaying band and auxiliary scores from the 2011 Arcadia Festival of Bands held in Arcadia, CA \cite{ARCADIA}.
%
%\begin{table}[h!]
%	\centering
%
%	\label{Band}
%	\begin{tabular}{|l|l|l|}
%		\hline
%		School Name & Band Score & Auxiliary Score \\ \hline
%		Rancho Bernardo & 96.15 & 89.15 \\ \hline
%		Mt. Carmel & 95.30 & 83.55 \\ \hline
%		Riverside King & 93.85 & 91.75 \\ \hline
%		Diamond Bar & 93.20 & 88.60 \\ \hline
%		El Dorado & 92.80 & 95.45 \\ \hline
%		Chino & 92.65 & 91.45 \\ \hline
%		Henry J. Kaiser & 92.60 & 87.55 \\ \hline
%		Glendora & 92.60 & 89.15 \\ \hline
%		Montebello & 90.50 & 82.70 \\ \hline
%		Mira Mesa & 89.65 & 91.50 \\ \hline
%	\end{tabular}
%	\caption{Scores from the 2011 Arcadia Festival of Bands.}
%\end{table}
%
%The table is sorted by band score. There is more text here to demonstrate how the template handles spacing between tables and body text. Also note how the table caption is in a smaller font size than the body text.
%
%\section{Equations}
%
%The following format is recommended to be used to display equations.
%
%%Make other examples.
%\begin{equation} \label{Equ.2.1}
%y=c_1\cos(t)+c_2\sin(t)
%\end{equation}
%\begin{equation} \label{Equ.2.2}
%e^{it}=\cos(t)+i\sin(t)
%\end{equation}
%
%Equation \ref{Equ.2.1} is the general solution to the differential equation $y''+y=0$. In the source code, the \textit{ref} command allows you to refer to an equation by a label you created. References must be made after the equation has been created; attempting to refer to an equation before it is defined results in a question mark placeholder. Some more sample equations are below. Notice the first set below is not numbered.
%%%
%\begin{align*}
%\log (x^n) &= \log (x \cdot x \cdot \ldots \cdot x) \\
%&= \log x + \log x + \ldots + \log x \\
%&= n \log x
%\end{align*}
%\begin{equation} \label{Equ.2.3}
%X^T X \mathbf{u} = X^T \mathbf{y}
%\end{equation}
%\begin{equation}\label{Equ.2.4}
%u(x, t) = \int_{-\infty}^{\infty} G(x, \tau) \exp\left(-\frac{(t-\tau)^2}{4kt}\right) \ d\tau
%\end{equation}
%\begin{gather}
%\mathcal{L}(f) = \int_{0}^{\infty} e^{-st} f(t) \ dt \\
%\begin{split} \label{Equ.2.5}
%\mathcal{F}(f) = \frac{1}{2\pi}\int_{-\infty}^{\infty} e^{i \omega x} f(x) \ dx
%\end{split}
%\end{gather}
%
%You can use labels to refer to equations you create. \ref{Equ.2.5} is the \textbf{Laplace transform} used extensively in differential equations. \ref{Equ.2.3} is the matrix representation of the \textbf{normal equations} used in least-squares regression.
%
%To have equations without labels appearing the right margin, simply add an asterisk to the name of the environment (equation, align, etc.) when making the declaration.
%
%
%\section{Theorems and Proofs: Examples}
%
%This section will show an example usage of the theorem and proof environments, typically used for mathematics students. To use these environments, you must have the package \textbf{amsthm} declared in the preamble of your document. For this template, this is already declared in the main file. You may choose to remove this declaration if your document will not make use of theorems and proofs.
%
%Theorems can be numbered, as the one below is, or you can force a different label to appear. For example, you can state the Bolzano-Weierstass theorem and have the names appear as the theorem label. See the examples below.
%
%Sometimes you may have a theorem with multiple parts or multiple conditions. You can use other list environments, such as enumerate, inside the theorem environment declared to list these conditions. The final example at the end of this block shows this with the Invertible Matrix Theorem, which has several equivalent statements.
%
%\newtheorem{thm}{Theorem}
%\begin{thm}
%	Suppose $f$ is of class $\mathcal{C}^1$ and $g$ is of class $\mathcal{C}^2$, and that the compact set $D$ and its boundary satisfy the hypotheses of Green's Theorem.  Then
%	\[ \iint \limits_D f\nabla^2 g \ dA = \oint_{\partial D} f(\nabla g) \cdot \mathbf{n} \ ds - \iint \limits_D \nabla f \cdot \nabla g \ dA . \]
%\end{thm}
%
%\begin{proof}
%	Begin with the integral of $f\nabla g \cdot n$ taken over the boundary of D.  By the second vector form of Green's Theorem,
%	\begin{align*}
%	\oint_{\partial D} f\nabla g \cdot n \ ds &= \iint \limits_D \nabla \cdot (f\nabla g) \ dA \\
%	&= \iint \limits_D f\nabla^2 g + \nabla f \cdot \nabla g \ dA.
%	\end{align*}
%	
%	Rearranging yields the desired.
%\end{proof}
%
%\begin{thm}[Bolzano-Weierstrass]
%	Every bounded real sequence has a convergent subsequence.
%\end{thm}
%
%\begin{thm}[Invertible Matrix Theorem\footnote{This is an incomplete list.}]
%	For any square matrix $A$ with $n$ rows and columns, the following are equivalent.
%	\begin{enumerate}
%		\item $A$ is invertible.
%		\item The equation $A\mathbf{x}=\mathbf{0}$ has only the trivial solution $\mathbf{x} = \mathbf{0}.$
%		\item For any nonzero $\mathbf{b}, \ A\mathbf{x} = \mathbf{b}$ has exactly one solution.
%		\item The columns of $A$ form a linearly independent set.
%		\item Zero is not an eigenvalue of $A$.
%		\item $A$ has full rank.
%		\item The determinant of $A$ is not zero.
%	\end{enumerate}
%\end{thm}
%
%There is currently no set format on how propositions and theorems should be laid out in the document. The idea is to remain consistent. It is best to not customize the appearance of theorems so that they can easily be distinguished from body text - just like figures, tables, and headings.
%
%\section{Another Table Example}
%For the sake of testing the appearance of the list of tables, a second table will be displayed here. This table displays a list of some major universities and their enrollments during fall 2015. This table is sorted in descending order of enrollment.
%%The savenotes environment, loaded from the footnote package
%%(which in turn is loaded from mdwtools)
%%allows you to use footnotes in tables, if needed.
%\begin{savenotes}
%\begin{table}[h!]
%	\centering
%	\label{my-label}
%	\begin{tabular}{|l|l|l|}
%		\hline
%		School & City and State & Fall 2015 Enrollment  \\ \hline
%		Texas A\&M University\footnote{Gig 'em!} & College Station, TX & 64,376  \\ \hline
%		Ohio State University\footnote{This number describes enrollments at the Columbus campus; enrollments at regional campuses in Lima, Mansfield, Marion, Newark, and Wooster are not counted.} & Columbus, OH & 58,322 \\ \hline
%		Iowa State University & Ames, IA & 36,001 \\ \hline
%		University of California, San Diego & La Jolla, CA & 33,735   \\ \hline
%		University of West Florida & Pensacola, FL & 12,798 \\ \hline
%		Massachusetts Institute of Technology & Cambridge, MA & 11,319   \\ \hline
%	\end{tabular}
%	\caption{Some major universities and their fall 2015 enrollments.}
%\end{table}
%\end{savenotes}
%
%Naturally, tables and footnotes do not go together. If you attempted to write a footnote inside a table, there will be nothing at the bottom of the page, yet the footnote marker will still appear. To remedy this, the \textit{footnote} package has been loaded from the \textit{mdwtools} package. Check your TeX distribution to see if \textit{mdwtools} is installed. See the source code for how this is implemented.
%
%Here are some blank floats.
%
%\begin{figure}[!h]
%	\caption{A blank float.}
%\end{figure}
%
%\begin{figure}[!h]
%	\caption{Another blank float.}
%\end{figure}