%%%%%%%%%%%%%%%%%%%%%%%%%%%%%%%%%%%%%%%%%%%%%%%%%%%
%
%  New template code for TAMU Theses and Dissertations starting Fall 2016.  
%
%
%  Author: Sean Zachary Roberson
%  Version 3.17.09
%  Last Updated: 9/21/2017
%
%%%%%%%%%%%%%%%%%%%%%%%%%%%%%%%%%%%%%%%%%%%%%%%%%%%


%%%%%%%%%%%%%%%%%%%%%%%%%%%%%%%%%%%%%%%%%%%%%%%%%%%%%%%%%%%%%%%%%%%%%%
%%             CONTRIBUTORS AND FUNDING SOURCES
%%%%%%%%%%%%%%%%%%%%%%%%%%%%%%%%%%%%%%%%%%%%%%%%%%%%%%%%%%%%%%%%%%%%%
\chapter*{CONTRIBUTORS AND FUNDING SOURCES}
\addcontentsline{toc}{chapter}{CONTRIBUTORS AND FUNDING SOURCES}  % Needs to be set to part, so the TOC doesn't add 'CHAPTER ' prefix in the TOC.

This work was sponsored by the Aerospace Vehicles Systems Institute (AVSI) through Authorization for Expenditure 76s1 (AFE 76s1). Project members contributed funding, equipment, and expertise without which the project would not have been completed. Project members included Honeywell*, Rockwell Collins*, Thales*, Garmin*, Airbus, Boeing, Embraer, Lufthansa, UTC, Zodiac, NASA, FAA, IATA. Starred project members contributed widely used commercial radio altimeters for testing.  

Thomas Meyerhoff from Airbus contributed the worst case landing scenario analysis covered in Section~\ref{subsub:worst_case}, and the summary of initial testing results shown in Figure~\ref{fig:initial_summary}. Dr. Chadi Geha provided the concept for the modified altimeter test setup shown in Figure~\ref{fig:Modified}. All other work presented in this thesis was conducted by the author. 

\pagebreak{}