%%%%%%%%%%%%%%%%%%%%%%%%%%%%%%%%%%%%%%%%%%%%%%%%%%%
%
%  New template code for TAMU Theses and Dissertations starting Fall 2016.  
%
%
%  Author: Sean Zachary Roberson
%  Version 3.17.09
%  Last Updated: 9/21/2017
%
%%%%%%%%%%%%%%%%%%%%%%%%%%%%%%%%%%%%%%%%%%%%%%%%%%%

%%%%%%%%%%%%%%%%%%%%%%%%%%%%%%%%%%%%%%%%%%%%%%%%%%%%%%%%%%%%%%%%%%%%%%
%%                           SECTION I
%%%%%%%%%%%%%%%%%%%%%%%%%%%%%%%%%%%%%%%%%%%%%%%%%%%%%%%%%%%%%%%%%%%%%


\pagestyle{plain} % No headers, just page numbers
\pagenumbering{arabic} % Arabic numerals
\setcounter{page}{1}


\chapter{\uppercase {Introduction}}


\section{Motivation}

Over the past two decades, wireless digital communication systems have become ubiquitous in the public life. As the technologies have become more proven, a broad range of players in the aerospace industry developed a significant interest in deploying these systems to electronics on airplanes. 

Avionics manufacturers are interested in the development, sale and deployment of sensors and devices in areas on a plane that were difficult or impossible to reach with wireless systems.  Examples might include placing sensors to monitor a landing gear or internal to an engine. Airframers, Aircraft OEMs and Airlines all have a vested interest in any development which could reduce the amount of copper wiring on planes, thus reducing weight and fuel costs. Regulators are interested in wireless avionics systems from a safety perspective. Critical avionics systems have redundant paths wired in in case of failure, and some or all of these redundancies may eventually be replaced  with wireless ones. This type of dissimilar redundancy is always appealing from a safety perspective. The example of an engine fire which destroys the physical connection to a controller (and so can't be shut off) comes to mind. 

[Add a better transition here]

\section{History} 

With these motivations spanning across the industry, various aerospace companies sponsored a series of working groups to implement wireless communications systems on aircraft. This systems used in this class of applications were dubbed Wireless Avionics Intra-Communications (WAIC) systems. WAIC related projects were sponsored and conducted through the Aerospace Vehicle Systems Institute (AVSI), which also directed this project. AVSI projects are funded through independent grants known as Authorizations For Expenditure (AFEs)

Three AVSI projects directly relate to WAIC: AFE 56, AFE 73, and AFE 76. AFE 56 studied the feasibility of potential WAIC systems  and investigated the suitability of various bands to WAIC applications. AFE 73 took the analysis done by AFE 56 and used the work to advocate to regulators for spectrum allocation for WAIC. 

\section{Project Goals}
This work was funded through AVSI and managed under AFE 76. The goal of this project was to perform a band sharing study for WAIC with radio altimeters, and to develop a protype for WAIC systems. The technical challenges in this project directly result from both the technical studies and the inherently political interactions with regulators performed under its predecessors. Because of this, a brief summary of the work done by the two preceding AVSI projects will be presented here, emphasizing the portions of each most relevant to this study. 

\section{WAIC Feasibility Study}
The WAIC Feeasibility study was conducted through AFE 56, and the results of this study were published in [1]. The report is summarized here for background with significant focus placed on the search for a suitable WAIC band. 

AFE 56 had three primary goals: 

\begin{itemize}
\item ''Identify, Characterize and prioritize the most significant obstacles currently impeding widespread use of wireless communication in flight-critical functions''

\item ''Evaluate the current aircraft RF certification process and suggest possible modifications or changes''

\item ''Identify the most promising avenues to certify reliable and robust wireless intra-aircraft data transmission''
\end{itemize}

Toward these ends, investigations were performed into the certification process, suitable spectrum bands, and security concerns related to the implementation of WAIC systems. 

\subsection{Certification}
Any equipment installed on an aircraft is subject to the regulatory certification process, which functions as a way for regulatory bodies to declare the equipment airworthy. All civilian and military aircraft are subjected to some form of this process. The AFE 56 working group surveyed the various standards imposed by the DoD, FAA, and ICAO (International Civil Aviation Organization), as well as international treaties governing the aviation indusrty. The committee took an in depth look at the flight clearance process in use at various agencies. 

The AFE 56 working group then looked at the specific challanges brought to the forefront by wireless systems. The primary concerns for potential WAIC  systems involved the sharing of spectrum with other legal occupants of the band,  as well as intentional and unintentional interferers. It was determined that a certification process for WAIC systems would need to account for and provide mitigation strategies for each of these various potential interferers to pass certification. Information security would also need to be guarunteed for critical systems. These considerations would drive the band selection process for WAIC. 

\subsection{The Search for a Suitable WAIC Band}
Prior to beginning the search for a suitable band, members of the project management committee held discussions with the FCC to gain insight on the regulator's perspective on allocations for potential WAIC systems. Several points of discussion were notable. When asked for clarification on how to classify these wireless sensor networks, the FCC staff ''were equally at a loss'' to AVSI engineers on the specifics of classifying WAIC services. Secondly, FCC staff reccommended AVSI pursue an international spectrum allocation before focusing on domestic rulemaking. 
	
	Finally, the FCC placed significant emphasis on the importance of  ''picking a winner'' as quickly as possible in the frequency selection process. This reccomendation was made in light of experience with previous international radio projects. American industry coordinated a global effort to upgrade the Weather Fax system which was delayed by more than two years and ultimately only partially successful. The FCC ultimately pinned these issues on the failure of American industry to ''\textit{socialize} their specific solution'' with key international players in the international spectrum allocation procees.  
 
Industrial, Scientific, and Medical or ISM bands are subjected to limtied regulations, and were quickly eliminated from consideration for WAIC devices. A wide variety of commercial devices already occupy this band, and these devices are allowed to radiate at relatively high powers. Because of the high operating powers, users are afforded no regulatory protection from hamful interference, a condition which would be unacceptable for the safety focused aerospace industry. For this reason the 915 MHz, 2.4 GHz, 5.8 GHz, 24 GHz, and 61 GHz bands were eliminated from consideration for WAIC devices. 


To find a suitable alternative, the committee stepped through both the US and international tables of frequency allocations. The committee rated alternatives according to two goals. The first was electromagnetic compatibility with wireless sensor applications . The second goal was that a suitable band already be allocated or have potential to be allocated in a manner compatible with WAIC desired properties. 

A series of criteria were used to rate the suitability of potential alternatives. A band already primarily allocated to an aeronatical service was considered beneficial from the political perspective of spectrum allocations. Benign co-primary users were considered essential. The less sensitive other occupants are to the minimal level of interference from on-aircraft wireless systems, the better. Bands which possessed common allocations across international regions were also considered beneficial, to ease the process of getting approval for WAIC use of the band. 

It was considered critical that WAIC systems be sufficiently isolated from ISM and unlicensed allottments. The relatively uncontrolled emmitters were considered a significant threat to on aircraft wireless. Isolation from terrestrial point to point systems was also considered crictical. These systems introduce the possibility of impinging extremely high radiated power levels onto aircraft that pass through. Although this risk is limited to low altitudes, it constitutes a significant safety hazard that can be avoided by the choice of band. A final consideration for allocations is isolation from Sattelite (Earth to Space) allocations. Uplink sites require significant RF power to maintain, and therefore consist of a safety hazard similar to point to point systems. 

\subsection{Candidate Bands}
Based on these criteria, the AFE 56 committee performed a review of major candidate bands for WAIC systems. The comittee provided a synopsis of relevant characteristics of each candidate band and rated the band according to it's suitability. AVSI performed this process with a goal of helping future working groups to prioritize future efforts at reserving spectrum allocation. 

\subsubsection{4.200 - 4.400 GHz}
At the time of AFE56, the 4.2-4.4 GHz band was exclusively allocated to radio altimiters. The committee found that technical hurdles in this band consisted primarily of coexisting with radio altimetry services. This coexistence could be accomplished through isolation (altimeter pulse signals do not occupy the full band, so a WAIC system could occupy the band edges). Despite this possibility, the committee anticipated proof of noninterference would be a significant hurldle for WAIC systems to overcome during the certification process. Proof of benign levels of interference would likely have to come in both theoretical forms and in the form of testing of a candidate WAIC system with real altimeter systems. These regulatory hurdles made the 4.200 - 4.400 GHz band relatively unattractive from the perspective of AFE 56. However, since the band was exclusively occupied by aeronautical services, it remained under consideration. 

\subsubsection {4.800 - 4.940 GHz}
The 4.800-4.940 band is allocated for federal FIXED and MOBILE Services (unclear what these are) as well as  a secondary allocation for radio astronomy. The committee recommended potentially avoiding the 4.825-4.825 portion of the band to avoid conflict with radio astronomy interests. 

\subsubsection{5.030-5.091 GHz}
This band is exclusively allocated to aviation, however it is primarily used for the Microwave Landing System (MLS). If it weren't for the presence of this critical system, the band would be an attractive candidate for WAIC. While band sharing workarounds are technically feasible, the critical nature of MLS would make regulatory and certification barriers an excessive burden on any WAIC implementation. 

\subsubsection{5.091-5.150 GHz Band} 
Like the previous band, this band is allocated to aviation interests for MLS. Unlike the previous band, this band has presently seen almost no use by MLS. Because of the lack  of use to date, the band has become known as the MLS expansion band. Due to the lack of use in aviation, the allocation has been modified temporarily to allow primary usage by Earth to space uplink site services. Since isolation from these services was specifically listed as a criteria for candidate bands, this was considered a major drawback. The committee was uncertain whether MLS would expand into this band or uplink sites would see continued usage past 2018, and the variety of interests in this band combined to create a potentially difficult situation to negotiate when petitioning ICAO for allocation. 

\subsubsection{5.150 - 5.250 GHz}
Similar to the MLS expansion band, this band has primry allocation to the Aeronautical radionavigation services which have never been used to date. Fixed satellite uplink services and international mobile users have co-primary allocation. While sharing the band with uplink services contradicts one of the criteria listed for ideal candidate WAIC bands, the AFE 56 working group felt that a band sharing program limiting the satellite services to a sub-band could be feasible. 


\subsubsection{5.350 - 5.460 GHz}
AFE 56 found that this band was allocated to a range of services on a co-primary basis. Earth Exploration Satellite, Space Research, Aeronautical Radionavigation, and Radiolocation services are included in this allocation among others. This band hosts weather radar systems, airborn radar systems operated by the U.S. military, terrestrial military radar systems, and ship born radar systems. Satellite and Space Research users gained access to this band relatively recently. While technically feasible the range of competing interests on this band made it unnactractive to the AFE 56 committee from the perspective of applying for an allocation. While acquiring a secondary allocation might be possible, fighting the opposition of a broad group of existing interests in this band indicated to the committee that other options might be better. 

\subsubsection{8.750 - 8.850 GHz}
Radiolocation and radionavigation services have primary allocations to this band on a global basis. In the past, this has been used for airborne Doppler radar sensing systems. The AFE 56 working group found this band to be an interesting option because of the decline in use of Doppler-based navigation in present-day aviation. However, in the U.S., doppler based systems also have an allocation to 13.25-13.4 GHz, so regulators have mandated that they must accept any interference from fixed military radar operating in 8.75 - 8.85 GHz. This mandate makes the band less attractive than it otherwise might have been. 

\subsubsection{13.25 - 13.40 GHz}
Similar to the pervious band, this band has a primary allocation to radionavigation services. While sattelite and space research users also have allocations to this band, the ITU modified their allocation to essentially have secondary status to radionavigation services. The attractiveness of this band is similar to the 8.75 GHz band in that the doppler radar navigation systems have declined in use over time. On top of this, the mandate giving preference to military radar systems in the previous band does not exist in th 13.25-13.40 GHz band. The committee found these factors attractive for a potential WAIC service. Existing primary allocation to aviation made this band a potential priority for further investigation.

\subsubsection{15.40 - 15.43 GHz and 15.63 - 15.70 GHz}
The project committee looked at these bands together because only 200 MHz separates the two and their allocations are identical. The bands were found to have exclusive allocation to Aeronautical radionavigation. Despite the advantage of a sole aviation allocation, the 200 MHz separating the bands was occupied by Fixed Sattelite Earth to Space uplink services, and fears of interference bleeding over from this service make these bands less viable for WAIC systems. 

\subsubsection{36.0 - 37.0 GHz}
The 36-37 GHz band was found to have primary allocations to passive Earth Exploration and space research sattelite applications. Fixed and Mobile services also possesed an allocation but the committee found that there was little use by these services. AFE 56 found that the radio astronomy community was conducting studies which would result in regulatory transmitter power limits for Fixed and Mobile users. This transmitter limit was considered beneficial for preventing interference to WAIC systems, and the committee considered this band a reasonable option for potential WAIC systems. 


\subsubsection{66.0 - 71.0 GHz}
This band has primary allocations to several different services, and is available for both federal and non-federal governement use. The committee found that the band was in use by the intelligence community for secure, short-range links and the military for intersattelite communication links. This band experiences the absorbtive affects of oxygen which are considered advantageous for systems which only need to work within the confines of a single aircraft.

\subsubsection{76.0 - 77.5 GHz}
 AFE 56 found that htis band has isolation similar to the previous band, although it is beginning to find use for short range radar sensors for vehichles. The committee found the band worth more consideration. 
 
\subsubsection{Summary of Candidate Bands}
 Need to get permission for this table or make my own figure summarizing results
 \begin{figure}
 \centering
 \includegraphics[scale=1]{Candidate_Bands.PNG}
 \caption{Figure from AFE56 final report}
 \label{fig:Candidates}
 \end{figure}
 
\subsection{Channel Modeling and Security}
Finally, the AFE56 committee surveyed two more obstacles to a finalized WAIC implementation. The committee looked at channel modeling for wireless sensor networks and  gave an overview of security concerns. 

Any implementation of WSN's on aircraft has the potential to be critical to the safety of flight. Because of this, the committee stressed the importance of  developing a validated channel model for the band and airframes in use. The channel models would allow for the incorporation of the phsyical propagation characteristics of the wireless signals in various aircraft and could be used to improve the reliability of real WAIC designs. Because of this, the comittee provided an overview of channel modeling efforts in their report and made recommendations for an approach to channel modeling efforts that might follow a new WAIC allocation. 

Lastly, the committee looked into the security concerns associated with WAIC systems. A report was comissioned through the University of Minnesota, which aimed to analyze the various potential threats to wireless networks on aircraft. The threat vectors considerd included physical layer attacks such as jamming, as well as higher layer attacks such as distributed denial of service and hacking risks. The report then looked at potential mitigation strategies for each of these threat vectors. The solutions listed were meant to be a comprehensive overview, but to acquire certification each device would have to provide a detailed overview of their implementation to the relevant certification authority. 

\subsection{Summary and Conclusions}
The AFE 56 project committee performed a feasibility study for wireless sensing network on aircraft. The comittee first looked at the existing path to certification for instrumentation, and came to the conclusion that this path would work for WSN's as well, provided that the applicant for certification perform the necessary extra step of explaining to the FAA the added risks of the wireless device and how these risks were mitigated. The committee then performed an in depth survey of potential bands for WSN use, summarizing the desireable characteristics possessed by any candidate band. The committee provided an in depth overview of the pros and cons of each serious candidate for WAIC allocation, a brief summary of which has been included in this report for reference. Finally, the  committee looked at potential channel modeling techniques and security concerns associated with wireless systems on aircraft and outlined how these would need to be addressed for a real WAIC implementation. 

The committee came to the conclusion that although there were numerous hurdles in the way of fully realizing WAIC systems, WAIC systems were feasible and these challenges could be overcome with industry expenditure and effort. The tasks necessary for WAIC implentation were as follows: 
\begin{itemize}
\item Acquire spectrum for WAIC use
\item Perform a band sharing study for WAIC and existing band occupants
\item Develop industry standards for channel modeling of airfraimes
\item Develop industry standards for addressing security concerns for wireless networks on aircraft
\item Work with regulators to develop a streamlined certification process for wireless sensing networks once these standards are developed
\end{itemize}
 
 With the feasibility study complete, the AVSI partners moved on to the task of acquiring spectrum for wireless networks on aircraft. 
 
\section{Spectrum Allocation for WAIC}
AVSI chose this band because... [Need a source]

The 2015 World Radio Confernce (WRC-15) made changes to the spectrum allocations in and around the radio altimeter (RA) band. New allocations for 5G systems in the 3.7 GHz (3600-4200 MHz) and 4.5 GHz (4400 - 4900 MHz)

\section{Overview of Radio Altimeter Functionality}

\subsection{Basic Overview and Applications}
The 4.2-4.4 GHz band was previously allocated exclusively to radio altimeters and transponder systems associated with altimetry. The altimeter functions to actively and continuously provide height measurements of an aircraft above the surface of the Earth. The highest degree of accuracy is expected in the approach, landing, and climb phases of flight. This accuracy must be maintained through all types of ground reflectivity. The height measured by an altimeter has a variety of uses in safety critical systems. The height functions as an input The Terrain Awareness Warning System, which gives the pilot a  ''Pull Up'' warning at a predetermined unsafe altitude and descent rate. The height from altimeters also functions as input for Collision Avoidance, Weather, Navigation, and Autopilot systems. Radio Altimeters are expected to operate in these functions through the lifetime of the Aircraft they are installed on, which results in Altimeters used in excess of 30 years.

\subsection{Calculating the Height From a Time Delay}
There are two primary types of altimeters in use today. The first are Frequency Modulated Carrier Wave (FMCW) Altimeters. FMCW altimeters use a transmitter and receiver with separate antennas. The signal from the transmit antenna travels to the ground, is reflected, and returns to the aircraft. Due to the constant propagation speed, the return time of the signal is proportional to the height of the aircraft above the surface. 
\begin{figure}
 \centering
 \includegraphics[scale=0.5]{FMCW_Example.PNG}
 \caption{Figure from ITU-R M.2059-0}
 \label{fig:FMCW}
\end{figure}

The signal travel time is based on the return of a signal of the same frequency as the transmit signal. One method for calculating the travel time of a signal involves taking the difference between the frequency of the return signal at the current time and the frequency of the transmit signal at the current time, $\Delta f$. As shown in Figure~\ref{fig:FMCW}, given a constant waveform, the return time of a signal is:

$$\Delta t = \frac{\Delta f}{df/dt}$$

Once $\Delta t$ is calculated, it the height can be determined using the speed of light: 

$$H = \frac{c}{2\Delta t} $$

While not relying on a frequency modulated waveform, pulsed radar altimeters use a series of discrete pulses to track the current height of the aircraft. The $\Delta t$ between two pulses is used to calculate the height in the same manner that an FMCW altimeter does. Thus, for any altimeter under test in a lab setting, the time delay of the signal between the Altimeter TX and RX must be simulated to provide an accurate representation of real conditions. 
\subsection{Attenuation of the Altimeter Signal in Free Space}
A signal traveling from Altimeter Transmit and back to receive passes through multiple different sources of gain and attenuation. There is attenuation from cable losses, gain from the TX antenna, free space path loss as the signal travels toward the ground, loss from the scattering of the signal by the ground, path loss of the return signal, a gain from the receive antenna, and finally the attenuation from return cable losses. The DO-155 standards specify the worst case signal attenuation for different heights, standardized antennas,   ground scattering environments, and standardized cable attenuations. Like the time delay, any Altimeter lab setup has to simulate the DO-155 attenuation for various heights to be realistic. 
\subsection{Conclusions}
Radio Altimeters are a safety critical system in any aircraft, the output of which is used by other important airborn systems. Altimeters use the time it takes a signal to travel to the ground and back to calculate the height of an aircraft off the ground, and must be able to pick up a return signal which has been attenuated significantly depending on the height. To test radio altimeters in a lab setting, both the time delay and attenuation experienced by a real signal must be simulated. 
